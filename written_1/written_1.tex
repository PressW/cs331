%% written_1.tex
%% V1.0
%% 2017/4/08
%% by Preston Wipf
%%*************************************************************************

\documentclass[journal]{IEEEtran}

\usepackage[letterpaper,left=0.75in,right=0.75in,%
  top=1.5in,bottom=0.75in]{geometry}
\usepackage{listings}
\usepackage{color}
\usepackage{hyperref}

\definecolor{codegreen}{rgb}{0,0.6,0}
\definecolor{codegray}{rgb}{0.5,0.5,0.5}
\definecolor{codepurple}{rgb}{0.58,0,0.82}
\definecolor{backcolor}{rgb}{0.95,0.95,0.92}

\lstdefinestyle{single}{
  backgroundcolor=\color{backcolor},
  breaklines=false,
  numbers=none,
  keepspaces=false
}

\lstdefinestyle{mystyle}{
  backgroundcolor=\color{backcolor},
  commentstyle=\color{codegreen},
  keywordstyle=\color{magenta},
  numberstyle=\color{codegray},
  stringstyle=\color{codepurple},
  basicstyle=\footnotesize,
  breakatwhitespace=false,
  breaklines=true,
  captionpos=b,
  keepspaces=true,
  numbers=left,
  numbersep=5pt,
  showspaces=true,
  tabsize=2
}

\hyphenation{op-tical net-works semi-conduc-tor}
\lstset{style=mystyle}

\begin{document}
\onecolumn

\title{Written Assignment 1}
\author{Preston~Wipf}%

\markboth{Introduction to Artificial Intelligence, No.~1, April~2017}%
{Shell \MakeLowercase{\textit{et al.}}: Written Assignment 1}

\maketitle

\section{Smart Home Assistant Analysis}

\subsection{PEAS Description}

\begin{itemize}
\item \textbf{Performance} - speed of response, accuracy of response, personalized responses
\item \textbf{Environment} - home owners (users), room, potential noise, internet, databases
\item \textbf{Actuators} - speakers, LED lighting
\item \textbf{Sensors} - microphone, ethernet, potential integrations/extensions (Nest thermostat)
\end{itemize}

\subsection{Environment Description}

\begin{itemize}
\item \textbf{Partially Observable:} Not only are the sensors only 
activated by keywords (``Alexa'' or ``Okay, Google''), but the 
sensors are subject to noise contamination in loud environments.
\item \textbf{Multi-agent:} I lean towards multi-agent because the 
person using the home assistant can be thought of as a cooperating 
agent who can structure their language (query) to maximize the speed 
and accuracy of the returned result.
\item \textbf{Stochastic:} Two contributing factors: uncertainty of 
an answer, perhaps like a subjective question - ``Who is the most 
exciting athlete of all time?'' as well as the uncertainty of the 
question; for example, what if a dog barks at the time a question 
is being asked. Should that be considered part of the query?
\item \textbf{Sequential:} You could argue each question is isolated 
from the next, but in this case I purport that each question cumulatively 
contributes to a profile of the individual in order to return responses 
best suited to the individual.
\item \textbf{Static:} Based on my understanding of these systems, 
once a request has been received, the device no longer accepts input 
until it has returned a result. 
\item \textbf{Discrete:} I go back and forth on this, but ultimately 
because the environment is static with respect to the actual task of 
mapping the query to a response. Meaning once the query is received, 
there is nothing changing the query or results with respect the time 
delta from the reception of the keyword to end of the response.
\end{itemize}

\subsection{Agent Suggestion}

\noindent \textbf{Utility-based agent:} It is not as simple as was the 
response correct, but how timely was it and to what degree does it fit 
the user who requested it.


\section{True/False}

\noindent \textit{f: Suppose an agent selects its action uniformly at 
random from the set of possible actions. There exists a deterministic 
task environment in which this agent is rational.} \\

\noindent \textbf{True:} Because the task environment includes the 
performance metric, it can be manipulated such that random action is indeed rational. \\

\noindent \textit{g; It is possible for a given agent to be perfectly 
rational in two distinct task environments.} \\

\noindent \textbf{True:} Assuming by distinct the prompt means at 
least one distinguishing feature between the two environments, then 
predicting the sum of the outcome of a dice roll where in one environment 
the die are weighted to come up 3 4 and the other environment has the 
die weighted 5 2. \\

\noindent \textit{h: Every agent is rational in an unobservable environment.} \\

\noindent \textbf{False:} If the environment is unobservable, the 
agent is only rational if the built-in knowledge base is rational in 
the environment. \\

\noindent \textit{i: A perfectly rational poker-playing agent never loses.} \\

\noindent \textbf{False:} The agent can make the best possible move based 
on the combination of cards it receives, but the cards are distributed at 
random and thus the best (perfectly rational) move can still result in a loss. \\


\section{Modified Vacuum Environment}

\noindent \textit{a: Can a simple reflex agent be perfectly rational for 
this environment?} \\

\noindent \textbf{No.} Because the simple reflex agent is only aware of 
the current precept, it has no knowledge of squares it has already visited 
and thus will revisit clean squares. \\

\noindent \textit{b: Can a simple reflex agent with a randomized agent 
function outperform a simple reflex agent with a deterministic agent 
function? Explain why or why not.} \\

\noindent \textbf{Yes.} Without the ability to perceive the environment 
/ boundaries, a deterministic agent function can enter an infinite loop 
against an obstacle whereas probability tells us the random agent will 
eventually break out from an obstacle. \\

\noindent \textit{c: Can a reflex agent with state outperform a simple 
reflex agent? Explain why or why not.} \\

\noindent \textbf{Yes.} Because it now has state, it can not only remember 
the locations it has visited, but also map the environment such that it can 
move between any two points in the most efficient manner (somewhat irrelevant 
in this single dimensional case of left and right). \\

\end{document}


