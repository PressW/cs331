%% program_1_report.tex
%% V1.0
%% 2017/4/20
%% by Preston Wipf
%%*************************************************************************

\documentclass[journal]{IEEEtran}

\usepackage[letterpaper,left=0.75in,right=0.75in,%
  top=1.5in,bottom=0.75in]{geometry}
\usepackage{listings}
\usepackage{color}
\usepackage{hyperref}
\usepackage{amsfonts}

\definecolor{codegreen}{rgb}{0,0.6,0}
\definecolor{codegray}{rgb}{0.5,0.5,0.5}
\definecolor{codepurple}{rgb}{0.58,0,0.82}
\definecolor{backcolor}{rgb}{0.95,0.95,0.92}

\lstdefinestyle{single}{
  backgroundcolor=\color{backcolor},
  breaklines=false,
  numbers=none,
  keepspaces=false
}

\lstdefinestyle{mystyle}{
  backgroundcolor=\color{backcolor},
  commentstyle=\color{codegreen},
  keywordstyle=\color{magenta},
  numberstyle=\color{codegray},
  stringstyle=\color{codepurple},
  basicstyle=\footnotesize,
  breakatwhitespace=false,
  breaklines=true,
  captionpos=b,
  keepspaces=true,
  numbers=left,
  numbersep=5pt,
  showspaces=true,
  tabsize=2
}

\hyphenation{op-tical net-works semi-conduc-tor}
\lstset{style=mystyle}

\begin{document}
\onecolumn

\title{Program 1 Report: Missionaries and Cannibals}
\author{Preston~Wipf~and~James~O'Neal}%

\markboth{Introduction to Artificial Intelligence, No.~2, April~2017}%
{Shell \MakeLowercase{\textit{et al.}}: Missionaries and Cannibals Report}

\maketitle

\section{Methodology}
\noindent I wanted to include a disclaimer here regarding my project. Initially, 
I was told by another member of the class that we could work together in a team
of 3 (unsafely assuming this had come with instructor permission). I was notified
by my former team on Monday that we could not work as a 3 man cell, and I was the odd 
man out. Tuesday I found a person I found out there was another person in a similar
situation and thus this project attempt, is a bit of a last minute scramble to form a
new team and complete the project from scratch so as to not violate the integrity of 
our former teams. In order to do so, we completely changed methodologies to help 
distinguish our approach / code from theirs, but ran into a lot of problems in doing so. 
Those methodologies will be fleshed out in the following subsections. \\

\subsection{General Approach}
\noindent Our general approach to this problem wanted to move away from the knowledge
base we were working from on our initial teams which were a class-based systems
where each instance of the class represented the state of the left bank, right bank
and the boat. The differentiation was an attempt to create the state as a simple matrix,
which involved not only managing the matrices across the program, but also
associating matrix to matrix relationships such as parent matrices. The result is a 
bit hacky. \medskip

\subsection{States}
\noindent As I mentioned the states are represented by a 2x3 matrix in which each 
row represents a bank and each column is cannibals, missionaries and boat location
(similar to how it is seen in the start and goal files). The implementation was a 
pandas dataframe which made some operations really convenient (referring to matrix
indexes by row and column name), and others extremely difficult (adding to a priority 
queue which requires a ``truth'' value which is ambiguous for a dataframe).\smallskip

\noindent One other point to mention is that the explored states are stored in a 
dictionary or hash table which is keyed on a string representing the state. So given
a matrix [[0,0,0], [3,3,1]], the key would be ``000331'' as this uniquely identifies
the state. \medskip

\subsection{Successors}
\noindent The purpose behind the approach to states above is mainly for the computation
of successors. With the state represented as a matrix, the five possible moves for 
each state can be created by simple matrix addition. So a state of [[0,0,0], [1,3,3]]
could be added to [[0,2,1], [0,-2,-1]] to generate and action of moving the boat
from one bank to the other with 2 cannibals in it. Further, each direction could be 
changed by simply multiplying the matrix by a constant factor of -1. \smallskip

\noindent Successors which pass the validation check are added to a queue which represents
the fringe or frontier. In this implementation queue is a parameter to the
\textit{find\_successors()} function as the queue type is dependent on the algorithm 
being executed. Last point of successors is a quick note that a successor is associated to
its parent state by giving it the key or hash of its parent state. \medskip 

\subsection{Validation}
\noindent As just mentioned, there are two main pieces of validation in the software:
validation of successors (ie is the proposed state legitimate) and validation of a
goal state. In the former case, it is a simple check to ensure there are no negative
values and that no missionaries will be eaten by cannibals. Note that there is no 
validation for the members in the boat as this logic is handled by the five possible 
actions listed in the problem description. While the latter state compares the number
of missionaries, number of cannibals and boat locations in a given state against 
the goal state.\medskip


\subsection{Algorithm Parameters}
\noindent Lastly, is the parameters. In this case each search algorithm takes only
two parameters, the start state and the goal state. The start state is put into
the fringe queue and the goal state is validated against on each iteration. For the
queue of each algorithm, it is initialized within the algorithm and its type is dependent
on which algorithm is executing it. For example, bfs uses a FIFO queue, dfs and iddfs
use LIFO queues and the A* algorithm uses a Priority queue. \smallskip

\noindent The A* algorithm also makes use of the \textit{hueristic()} function
which takes a potential successor and the goal state as inputs and calculates the
hueristic of the move to the successor state as the sum of the difference between
the goal state's left bank and the successors state's left bank divided by
the size of the boat (2). In other words, it is the number of people yet to make 
it to the destination bank divided by the number of people who can go in a single
trip. This heuristic was accepted because ofdue to its admissibility. Because in each
instance the boat requires at least one person to bring it back to the opposing
bank, the rate at which people are moved between banks is one per trip. \medskip



\section{Results}
\begin{center}
 \begin{tabular}{|p{2cm} p{2cm} p{2cm} p{2cm}|} 
 \hline
 Algorithm & Test 1 & Test 2 & Test 3 \\ [0.5ex] 
 \hline\hline
 BFS & (12,28) & (34,184) & (378,8974) \\ 
 \hline
 DFS & (12,16) & (48,94) & (1798,4599) \\
 \hline
 IDDFS & (12,99) & (48,3590) & (???, $\infty$) \\
 \hline
 A* & (12,15) & (34,112) & (378,3851) \\ [1ex] 
 \hline
\end{tabular}
\end{center}
\begin{center}
Key: (\textit{solution\_depth, num\_nodes\_expanded}) \\ \medskip
\end{center}


\section{Discussion}
\noindent This is a difficult section to reconcile given the state of the program. 
Going in, the assumptions were that A* would be the superior algorithm and that we
would see relatively poor, but similar performance from DFS and BFS. Given that
IDDFS has a space complexity of \textit{O(b\^(d)} and that it executes each layer
\textit{m - (d - 1)} times where m is the max depth and d is the depth of the layer,
we expected it to use more space than the other algorithms. It was also expected for 
DFS to use less space than both BFS and IDDFS, but we also expected to find that DFS 
would demonstrate the lack of completeness of BFS, IDDFS and A*. \smallskip

\noindent One problem that has not been resolved as of this report is an intermittent
issue that results in an infinite loop or truncated solution path in the larger trees.
We have yet to isolate the root of the problem, but it is suspected that the problem
lies somewhere in the hacky class-less state association. \\



\section{Conclusion}
\noindent Our conclusion here is dependent on not only the results we have seen from previous
work, but also our previous understanding of the tradeoffs of the different algorithms. That said, given an
admissible, well-thought out heuristic, A* search is the best search algorithm. It
performs the best in terms of the number of nodes expanded, and returns a solution
depth that is on par or better than the rest of the algorithms. The results returned
from DFS were, as expected, lacking both in optimality and completeness while IDDFS
returned the optimal path, but in an exponential time and space complexity that far exceeded
the others. I don't think there is anything surprising here from the conclusions. The
results more reinforce previous understanding of the algorithms than present any sort of
novel view of their performance. \\


\end{document}

